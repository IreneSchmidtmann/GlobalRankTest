%\documentclass{beamer}
\documentclass[xcolor=pdftex,dvipsnames,table]{beamer}\usepackage[]{graphicx}\usepackage[]{color}
%% maxwidth is the original width if it is less than linewidth
%% otherwise use linewidth (to make sure the graphics do not exceed the margin)
\makeatletter
\def\maxwidth{ %
  \ifdim\Gin@nat@width>\linewidth
    \linewidth
  \else
    \Gin@nat@width
  \fi
}
\makeatother

\definecolor{fgcolor}{rgb}{0.345, 0.345, 0.345}
\newcommand{\hlnum}[1]{\textcolor[rgb]{0.686,0.059,0.569}{#1}}%
\newcommand{\hlstr}[1]{\textcolor[rgb]{0.192,0.494,0.8}{#1}}%
\newcommand{\hlcom}[1]{\textcolor[rgb]{0.678,0.584,0.686}{\textit{#1}}}%
\newcommand{\hlopt}[1]{\textcolor[rgb]{0,0,0}{#1}}%
\newcommand{\hlstd}[1]{\textcolor[rgb]{0.345,0.345,0.345}{#1}}%
\newcommand{\hlkwa}[1]{\textcolor[rgb]{0.161,0.373,0.58}{\textbf{#1}}}%
\newcommand{\hlkwb}[1]{\textcolor[rgb]{0.69,0.353,0.396}{#1}}%
\newcommand{\hlkwc}[1]{\textcolor[rgb]{0.333,0.667,0.333}{#1}}%
\newcommand{\hlkwd}[1]{\textcolor[rgb]{0.737,0.353,0.396}{\textbf{#1}}}%
\let\hlipl\hlkwb

\usepackage{framed}
\makeatletter
\newenvironment{kframe}{%
 \def\at@end@of@kframe{}%
 \ifinner\ifhmode%
  \def\at@end@of@kframe{\end{minipage}}%
  \begin{minipage}{\columnwidth}%
 \fi\fi%
 \def\FrameCommand##1{\hskip\@totalleftmargin \hskip-\fboxsep
 \colorbox{shadecolor}{##1}\hskip-\fboxsep
     % There is no \\@totalrightmargin, so:
     \hskip-\linewidth \hskip-\@totalleftmargin \hskip\columnwidth}%
 \MakeFramed {\advance\hsize-\width
   \@totalleftmargin\z@ \linewidth\hsize
   \@setminipage}}%
 {\par\unskip\endMakeFramed%
 \at@end@of@kframe}
\makeatother

\definecolor{shadecolor}{rgb}{.97, .97, .97}
\definecolor{messagecolor}{rgb}{0, 0, 0}
\definecolor{warningcolor}{rgb}{1, 0, 1}
\definecolor{errorcolor}{rgb}{1, 0, 0}
\newenvironment{knitrout}{}{} % an empty environment to be redefined in TeX

\usepackage{alltt}
\usepackage[english,ngerman]{babel}
\usepackage{etex}
\usepackage{amsmath,amsfonts, bbm, , nicefrac, amssymb}
\usepackage{nccfoots}
\usepackage{cmap}
\usepackage{lmodern}
\usepackage[latin1]{inputenc}
\usepackage[T1]{fontenc}
\usepackage{colortbl,booktabs}
\usepackage{microtype}
\usepackage{subfigure}
%\usepackage{tikz} % pgf wird mitgeladen
%\usetikzlibrary{arrows}
%\usefonttheme[onlymath]{serif}

\xdefinecolor{darkblue}{rgb}{0.2,0.2,0.7}
\xdefinecolor{bleu}{rgb}{0.7,0,0}
\xdefinecolor{mauve}{rgb}{0.7,0,0.6}
\xdefinecolor{darkgreen}{rgb}{0.1,0.6,0.3}

\usepackage{xspace}
\newcommand{\zB}{z.\,B.\xspace}

%%%%%%%%%% meine Vorlage %%%%%%%%%%%%%%%%%%%%%%%%%
%\mode<presentation>{
%\usetheme{Rochester}
%\usecolortheme{dolphin}% seahorse} %
%\useinnertheme{rounded}
%}
%\setbeamercolor{title in head/foot}{fg=black, bg=blue!25}
%\setbeamertemplate{footline}                                              % Inhalt der Fu?zeile
%{
  %\begin{beamercolorbox}[wd=\paperwidth,ht=2.3ex,dp=1.5ex,leftskip=1.1em,rightskip=1em]{title in head/foot}
%  \usebeamerfont{title in head/foot}
  %\insertshorttitle\hfill\insertframenumber/23%\insertpresentationendpage - funktioniert irgendwie nicht wegen \only...
%  \end{beamercolorbox}
%}
%\beamertemplatetransparentcovereddynamic 
%\beamertemplateballitem
%\beamertemplatenavigationsymbolsempty
%\setbeamercolor{alerted text}{fg=red!70!black}
%\setbeameroption{show notes}

%Die zu verwendenden Farben ergeben sich aus dem Design Manual:
%Rot: 	   R: 193; G: 0; B: 43
%Dunkelblau: R: 0; G: 60; B: 118
%Hellblau: 	   R: 128; G: 161; B: 201
 \definecolor{UniBlue}{RGB}{128,161,201}
 \definecolor{UniDarkBlue}{RGB}{0,60,118}
 \definecolor{UniRed}{RGB}{193,0,43}
%	\setbeamercolor{title}{fg=UniBlue}
%%%%%%%%%%%%%%% Ulrikes Vorlage %%%%%%%%%%%%%%%%%%%%%%
\setbeamercolor{frametitle}{bg=UniBlue, fg=UniDarkBlue}%{bg=blue!12}
\setbeamercolor{title}{bg=UniBlue, fg=UniDarkBlue}%{bg=blue!12}
\setbeamercolor{block}{bg=UniBlue, fg=UniDarkBlue}%{bg=blue!12}
\setbeamercolor{framefootline}{bg=UniBlue, fg=UniDarkBlue}%{bg=blue!12}
\setbeamercovered{transparent}
\setbeamercolor{item}{fg=UniRed}
\setbeamertemplate{itemize item}[square]
%\setbeamertemplate{footline}[page number]
%\useinnertheme{circles}\useinnertheme{rounded}
%\setbeamercovered{transparent}
\beamertemplatenavigationsymbolsempty

\setbeamertemplate{footline}                                              % Inhalt der Fu?zeile
{
  \begin{beamercolorbox}[wd=\paperwidth,ht=2.3ex,dp=1.5ex,leftskip=1.1em,rightskip=1em]{title in head/foot}
  \usebeamerfont{title in head/foot}
  \insertshorttitle\hfill\insertframenumber%/\insertpresentationendpage
  \end{beamercolorbox}
}
\setbeamercolor{title in head/foot}{bg=UniBlue, fg=UniDarkBlue}%{bg=blue!12,fg=darkblue}
%%%%%%%%%%%%%%%%%%%%%%%%%%%%%%%%%%%%%%%%%%%%%

%\beamertemplateballitem

\setbeamercolor{alerted text}{fg=UniRed}%{fg=red!70!black}

\title{Power of the Wilcoxon-Mann-Whitney test for non-inferiority in the presence 
of death-censored observations} %slide 1
\author{Irene Schmidtmann\inst{1} \and 
Stavros Konstantinides\inst{2} \and 
Harald Binder\inst{3}}

\institute[University Medical Center Johannes-Gutenberg-University Mainz] % (optional, but mostly needed)
{
  \inst{1}%
  Institute for Medical Biostatistics, Epidemiology and Informatics (IMBEI)
  \and
  \inst{2}%
  Center for Thrombosis and Hemostasis Mainz (CTH)\\
University Medical Center Johannes-Gutenberg-University Mainz
\and
\inst{3}%
Institute for Medical Biometry and Statistics  \\
University of Freiburg}

\date{September 1st, 2017}

\titlegraphic{\includegraphics[width=8cm]{Universitaetsmedizin.jpg}}%uni-imbei.jpg}}

%%%%%%%%%%%%%%%%%%%%%%%%%%%%%%%%%%%%%%%%%%%%%%%%%%%%%%%%%%%%%%%%%%%%%%%%%%%%%%%%%%%%%%%%%%%%%%%%%%%%%%%%%%%%%%%%%%%%%%%%
%%%%%%%%%%%%%%%%%%%%%%%%%%%%%%%%%%%%%%%%%%%%%%%%%%%%%%%%%%%%%%%%%%%%%%%%%%%%%%%%%%%%%%%%%%%%%%%%%%%%%%%%%%%%%%%%%%%%%%%%
\IfFileExists{upquote.sty}{\usepackage{upquote}}{}
\begin{document}
%\linespread{1.5}

%\frame{\titlepage} 

\begin{frame}[plain] %title
\titlepage
\end{frame}

\begin{frame} %slide 2
\frametitle{The clinical problem}
Aim:\\
Comparing ultrasound accelerated, catheter-directed, low-dose thrombolysis (USCDT) in patients with high risk pulmonary embolism against standard thrombolysis
\begin{itemize}
	\item Primary Endpoint: reduction of RV/LV ratio 24 hours after randomization
	\item Testing for non-inferiority as new treatment
	\begin{itemize}
		\item is believed to be associated with lower bleeding risk than standard thrombolysis
		\item but not necessarily superior wrt primary endpoint
	\end{itemize}
	\item Problem: There is a non-negligible risk of death within 24 hours
	\begin{itemize}
		\item Excluding patients who die is not advisable
		\begin{itemize}
			\item Contradicts ITT principle
			\item May introduce bias
			\item Reduces Power
		\end{itemize}
	\end{itemize}
\end{itemize}
\end{frame}

\begin{frame} %slide 3
\frametitle{The situation in statistical terms}
	\begin{itemize}
		\item Quantitative endpoint determined at time $\tau$ \\
    $X_{01}, \ldots, X_{0n_0}$  values of quantitative endpoint in group 0 (reference)\\
    $X_{11}, \ldots, X_{1n_1}$  values of quantitative endpoint in group 1 (treatment) \\
		\item Events and event times \\
		$T_{i1}, \ldots, T_{in_i}$  event times in group $i$ \\
    $D_{ik} = \begin{cases} 1 & \mbox{if } T_{ik} < \tau \\
                            0 & \mbox{otherwise} 
              \end{cases}$ \\
		$q_i = S_i(\tau) $  survival probability in group $i$ at time $\tau$ \\
		$p_i = 1 - q_i = P(D_{ik} = 1)$  cumulative mortality in group $i$ until time $\tau$ \\
		\item If $D_{ik} = 1, X_{ik}$ cannot be observed
\end{itemize}
\end{frame}

\begin{frame} %slide 4
\frametitle{A possible solution}
	\begin{itemize}
		\item Combine quantiative endpoint and time to event outcome by using global ranks
		\item Assumptions
		\begin{itemize} 
		  \item High values of quantitative variable are beneficial
		  \item Death is worse than any quantitative outcome
		  \item Early death is worse than later death.
	\end{itemize}
	\item Introduce new variable 
	     $\tilde{X_{ik}} = D_{ik}(\eta  + T_{ik}) + (1 - D_{ik})X_{ik}$ with \\
	     $\eta = min(X_{01}, \ldots, X_{0n_0}, X_{11}, \ldots, X_{1n_1}) - 1 - \tau$
	\item Use $\tilde{X_{ik}}$ to determine ranks and to compute the Wilcoxon-Mann-Whitney test statistic
\end{itemize}
\end{frame}





\begin{frame}[fragile]{First Test}  %slide 3
\begin{knitrout}
\definecolor{shadecolor}{rgb}{0.969, 0.969, 0.969}\color{fgcolor}
\includegraphics[width=0.4\textwidth,height=0.4\textheight]{figure/Figure1-1} 

\end{knitrout}
\end{frame}


\begin{frame}[fragile]{First Test}

OK, let's get started with just some text:



BTW, the first element of \texttt{x} is 0.1449583. (Did you notice
the use of\texttt{ \textbackslash{}Sexpr\{\}}?)
\end{frame}

\section{Second Test}
\begin{frame}[fragile]{Second Test}

Text is nice but let's see what happens if we make a couple of plots
in our chunk:

\begin{columns}
    \begin{column}{.5\linewidth}
\begin{knitrout}
\definecolor{shadecolor}{rgb}{0.969, 0.969, 0.969}\color{fgcolor}
\includegraphics[width=4cm,height=4cm]{figure/boring-plots-part1-1} 

\end{knitrout}
    \end{column}
    \begin{column}{.5\linewidth}
\begin{knitrout}
\definecolor{shadecolor}{rgb}{0.969, 0.969, 0.969}\color{fgcolor}
\includegraphics[width=4cm,height=4cm]{figure/boring-plots-pat2-1} 

\end{knitrout}
    \end{column}
  \end{columns}
\end{frame}

\begin{frame}[fragile]{Third Test}
\begin{knitrout}
\definecolor{shadecolor}{rgb}{0.969, 0.969, 0.969}\color{fgcolor}\begin{kframe}
\begin{alltt}
\hlkwd{par}\hlstd{(}\hlkwc{las}\hlstd{=}\hlnum{1}\hlstd{,}\hlkwc{mar}\hlstd{=}\hlkwd{c}\hlstd{(}\hlnum{4}\hlstd{,}\hlnum{4}\hlstd{,}\hlnum{.1}\hlstd{,}\hlnum{.1}\hlstd{))}  \hlcom{# tick labels direction}
\hlkwd{boxplot}\hlstd{(x)}
\hlkwd{hist}\hlstd{(x,}\hlkwc{main}\hlstd{=}\hlstr{''}\hlstd{,}\hlkwc{col}\hlstd{=}\hlstr{"blue"}\hlstd{,}\hlkwc{probability}\hlstd{=}\hlnum{TRUE}\hlstd{)}
\hlkwd{lines}\hlstd{(}\hlkwd{density}\hlstd{(x),}\hlkwc{col}\hlstd{=}\hlstr{"red"}\hlstd{)}
\end{alltt}
\end{kframe}
\includegraphics[width=0.45\textwidth]{figure/boring-plots-1} 
\includegraphics[width=0.45\textwidth]{figure/boring-plots-2} 

\end{knitrout}
\end{frame}

\end{document}
