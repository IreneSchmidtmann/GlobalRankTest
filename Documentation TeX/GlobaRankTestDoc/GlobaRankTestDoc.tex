\left.%\documentclass{beamer}
\documentclass[xcolor=pdftex,dvipsnames,table]{beamer}
\usepackage[ngerman]{babel}
\usepackage[latin1]{inputenc}
\usepackage{amsmath,amsfonts,amssymb}
%\usepackage{colortbl}

\definecolor{lightgray}{gray}{0.9}


\definecolor{UniBlue}{RGB}{128,161,201}
\definecolor{UniDarkBlue}{RGB}{0,60,118}
\definecolor{UniRed}{RGB}{193,0,43}

\title{Non-inferiority Global Rank Test} %slide 1
\author{Irene Schmidtmann}
\date{March 25th, 2017}
\begin{document}
\maketitle
\begin{frame} %slide 2
\frametitle{Purpose}
Determining $E(U)$ and $var(U)$ according to the formula from Matsouaka and Betensky
\end{frame}

\begin{frame} %slide 3
\frametitle{Notation - General}
%
\begin{tabular}{|lp{8cm}|lp{20cm}}\hline
Symbol               & Description \\
\hline
$i=0,1$              & denotes group, \\
                     & $i=0$ denotes reference group, \\
										 & $i=1$ denotes treatment group \\
$n_i$                & sample size group in $i$ \\
$n = n_0 + n_1$      & total sample size \\
$\tau$               & time at which quantitative endpoint is determined \\
\hline
\end{tabular}

Assuming exponential distributions for times to event and normal distributions with common variance for quantitative endpoint.

\end{frame}
%
\begin{frame} %slide 4
\frametitle{Notatation - Survival endpoint}
\begin{tabular}{|lp{8cm}|lp{20cm}}\hline
Symbol          & Description \\
\hline
$t_{0k}$        & time to event for k-th individal in reference group \\
$t_{1l}$        & time to event for l-th individal in treatment group \\
$f_{i}(t)$      & probability density function of time to event random variable in group $i$ \\
$F_{i}(t)$      & distribution function of time to event random variable in group $i$ \\
$S_{i}(t)$      & survivor function of time to event random variable in group $i$ \\
$p_i$           & mortality in group $i$ at time $\tau$ \\
$q_i= 1 - p_i$  & survival probability in group $i$ at time $\tau$ \\
$d_{0k}$        & event indicator for k-th individal in reference group, i.e. $d_{0k}=1$ if $t_{0k} < \tau$, 0 otherwise \\
$d_{1l}$        & event indicator for l-th individal in treatment group, i.e. $d_{1l}=1$ if $t_{1l} < \tau$, 0 otherwise \\
$\lambda_i=\frac{-\log(q_i)}{\tau}$       & hazard in group $i$ \\
$\mathit{HR}=\frac{\lambda_1}{\lambda_0}$ & hazard ratio \\
\hline

\end{tabular}
\end{frame}
%
\begin{frame} %slide 5
\frametitle{Notation - Quantitative endpoint}
\begin{tabular}{|lp{8cm}|lp{20cm}}\hline
Symbol        & Description \\
\hline
$x_{0k}$      & value of quantitative endpoint for k-th individal in reference group \\
$x_{1l}$      & value of quantitative endpoint for l-th individal in treatment group \\
$g_{i}(x)$    & probability density function of  quantitative endpoint in group $i$ \\
$G_{i}(x)$    & distribution function of quantitative endpoint in group $i$ \\
$\varphi(x)$  & probability density function of the standard normal distribution \\
$\Phi(x)$     & distribution function of the standard normal distribution \\
$\mu_i$       & mean in group $i$ \\
$\sigma$      & common standard deviation \\
$\delta = \mu_1 - \mu_0$       & difference between group means \\
$\frac{\mu_1 - \mu_0}{\sigma}$ & effect size \\
\hline

\end{tabular}
\end{frame}
%
%
\begin{frame} %slide 6
\frametitle{Formula by Matsouaka and Betensky}
Matsouaka and Betensky derive a formula for mean and variance for the global rank test statistic U with the definitions for $\tilde{X_{0k}}$ and $\tilde{X_{1l}}$ from the paper.\\

$U = \frac{1}{n_0 n_1}\sum_{k=1}^{n_0}\sum_{l=1}^{n_1}I(\tilde{X_{0k}} < \tilde{X_{1l}})$ \\

Then expectation and variance are (not just for the case of identical distributions in both groups) given by:

$\mu_U = E(U) = p_0 p_1 \pi_{t1} + p_0 q_1 + q_0 q_1 \pi_{x1} = \pi_{U1}$

$\sigma_U^2 = var(U) = (n_0 n_1)^{-1} (\pi_{U1} (1 - \pi_{U1}) +
                                         (n_0 - 1) (\pi_{U2} - \pi_{U1}^2) + 
                                         (n_1 - 1) (\pi_{U3} - \pi_{U1}^2))$
where

$\pi_{U1} = p_0 p_1 \pi_{t1} + p_0 q_1 + q_0 q_1 \pi_{x1}$

$\pi_{U2} = p_0^2 q_1 + p_0^2 p_1 \pi_{t2} + 2 p_0 q_0 q_1 \pi_{x1} + q_0^2 q_1 \pi_{x2}$

$\pi_{U3} = p_0 q_1^2 + p_0 p_1^2 \pi_{t3} + 2 p_0 p_1 q_1 \pi_{t1} + q_0 q_1^2 \pi_{x3}$

\end{frame}
%
\begin{frame} %slide 7
\frametitle{Formula by Matsouaka and Betensky continued}
and

$\pi_{t1} = P(t_{0k} < t_{1l} | d_{0k} = d_{1l} = 1)$

$\pi_{t2} = P(t_{0k} < t_{1l}, t_{0k'} < t_{1l} | d_{0k} = d_{0k'} = d_{1l} = 1)$

$\pi_{t3} = P(t_{0k} < t_{1l}, t_{0k} < t_{1l'} | d_{0k} = d_{1l} = d_{1l'} =1)$

$\pi_{x1} = P(x_{0k} < x_{1l})$

$\pi_{x2} = P(x_{0k} < x_{1l}, x_{0k'} < x_{1l})$

$\pi_{x3} = P(x_{0k} < x_{1l}, x_{0k} < x_{1l'})$
\end{frame}


\begin{frame} %slide 8
\frametitle{Terms in detail - $\pi_{t1}$}
\begin{eqnarray*}
  \pi_{t1} & = & \frac{P(t_{0k} < t_{1l} < \tau)}{P(t_{0k} < \tau)P(t_{1l} < \tau)}
= \frac{1}{p_0 p_1}\int_0^\tau f_0(u)\int_u^\tau f_1(v)dv du  \\
& = &  \frac{1}{p_0 p_1}\int_0^\tau f_0(u)\Big( F_1(\tau) - F_1(u)\Big)du \\
& = & \frac{1}{p_0 p_1}\Big( p_0 p_1 - \int_0^\tau f_0(u)F_1(u) du \Big) \\
\\
\end{eqnarray*}
\end{frame}

\begin{frame} %slide 9
\frametitle{Terms in detail - $\pi_{t2}$}
\begin{eqnarray*}
\pi_{t2} & = &\frac{P(max(t_{0k}, t_{0k'}) < t_{1l} < \tau)}{P(t_{0k} < \tau)P(t_{0k'} < \tau)P(t_{1l} < \tau)} \\
& = & \frac{1}{p_0^2 p_1} \int_0^\tau 2f_0(u) F_0(u)\int_u^\tau f_1(v)dv du \\
& = & \frac{1}{p_0^2 p_1} \int_0^\tau 2f_0(u) F_0(u) (F_1(\tau) - F_1(u)) du  \\
& = &  \frac{1}{p_0^2 p_1} \Big(p_1 \int_0^\tau 2 f_0(u) F_0(u) du - \int_0^\tau 2 f_0(u) F_0(u) F_1(u) du \Big) \\
& = & \frac{1}{p_0^2 p_1} \Big(p_1 F_0^2(u)|_0^\tau - F_0^2(u) F_1(u)|_0^\tau + \int_0^\tau F_0^2(u) f_1(u)du) \Big) \\
& = & \frac{1}{p_0^2 p_1} \int_0^\tau F_0^2(u) f_1(u)du \\
\\
\end{eqnarray*}
\end{frame}


\begin{frame} %slide 10
\frametitle{Terms in detail - $\pi_{t3}$}
\begin{eqnarray*}
\pi_{t3} & = &\frac{P(t_{0k}  < t_{1l} < t_{1l'} < \tau) + P(t_{0k}  < t_{1l'} < t_{1l} < \tau)}{P(t_{0k} < \tau)P(t_{1l} < \tau)P(t_{1l'} < \tau)} \\
& = & \frac{2 P(t_{0k}  < t_{1l} < t_{1l'} < \tau)}{P(t_{0k} < \tau)P(t_{1l} < \tau)P(t_{1l'} < \tau)} \\
& = & \frac{2}{p_0 p_1^2}\int_0^\tau f_0(u) \int_u^\tau f_1(v) \int_v^\tau f_1(w) dw dv du \\
& = & \frac{2}{p_0 p_1^2}\int_0^\tau f_0(u) \int_u^\tau f_1(v) (F_1(\tau) - F_1(v)) dv du \\
& = & \frac{2}{p_0 p_1^2}\int_0^\tau f_0(u) \Big(p_1 (F_1(\tau) - F_1(u)) - \frac{1}{2}(F_1^2(\tau) - F_1^2(u)) \Big)du \\
& = & \frac{1}{p_0 p_1^2}\Big(p_0 p_1^2 - 2 p_1 \int_0^\tau f_0(u) F_1(u) du +  \int_0^\tau f_0(u) F_1^2(u) du \Big) \\
\\
\end{eqnarray*}
\end{frame}

\begin{frame} %slide 11
\frametitle{Distribution assumptions about $\pi_{t1}, \pi_{t2}, \pi_{t3}$}
Now make use of the fact that $t_{ik} \sim exp(\lambda_i)$, i. e. $f_i(t) = \lambda_i e^{-\lambda_i t}$ and 
$F_i(t) = 1- e^{-\lambda_i t}$.
\end{frame}

\begin{frame} %slide 12
\frametitle{$\pi_{t1}$}
\begin{eqnarray*} 
\pi_{t1} & = & \frac{1}{p_0 p_1}\Big( p_0 p_1 - \int_0^\tau \lambda_0 e^{-\lambda_0 u} (1- e^{-\lambda_1 u})du \Big) \\
& = & \frac{1}{p_0 p_1}\Big( p_0 p_1 - p_0 - \frac{\lambda_0}{\lambda_0 + \lambda_1} ( e^{-(\lambda_0 + \lambda_1)\tau}  - 1) \Big) \\
& = & \frac{1}{p_0 p_1}\Big( p_0 p_1 - p_0 - \frac{1}{1 + \mathit{HR}} ( q_0 q_1  - 1) \Big) \\
& = & \frac{1}{p_0 p_1}\Big( p_0 p_1 - p_0 - \frac{1}{1 + \mathit{HR}} ( p_0 p_1  - p_0 - p_1)\Big) \\
& = & \frac{1}{p_0 p_1}\Big(p_0 (p_1 - 1)(1-\frac{1}{1+\mathit{HR}}) + \frac{1}{1+\mathit{HR}} p_1 \Big) \\
& = & \frac{1}{p_0 p_1(1+\mathit{HR})}(p_0 (p_1 - 1)\mathit{HR} + p_1) \\
\\
\end{eqnarray*}
\end{frame}

\begin{frame} %slide 13
\frametitle{$\pi_{t2}$}
\begin{eqnarray*} 
\pi_{t2} & = & \frac{1}{p_0^2 p_1}\int_0^\tau (1 - e^{-\lambda_0 u})^2 \lambda_1 e^{-\lambda_1 u} du \\
& = & \frac{1}{p_0^2 p_1}\Big( \int_0^\tau \lambda_1 e^{-\lambda_0 u} du - 2 \int_0^\tau \lambda_1 e^{-(\lambda_0 + \lambda_1) u} du \\
&   & + \int_0^\tau \lambda_1 e^{-(2\lambda_0 + \lambda_1) u} du\Big) \\
& = & \frac{1}{p_0^2 p_1}\Big(p_1 + \frac{2\lambda_1}{\lambda_0 + \lambda_1} (e^{-(\lambda_0 + \lambda_1)\tau} - 1) \\
&   & - \frac{\lambda_1}{2\lambda_0 + \lambda_1} (e^{-(2\lambda_0 + \lambda_1)\tau} - 1) \Big) \\
& = & \frac{1}{p_0^2 p_1}\Big(p_1 - \frac{2\lambda_1}{\lambda_0 + \lambda_1} (1 - q_0 q_1) + \frac{\lambda_1}{2\lambda_0 + \lambda_1}(1 - q_0^2 q_1) \Big) \\
& = & \frac{1}{p_0^2 p_1}\Big(p_1 - \frac{2 \mathit{HR}}{1 + \mathit{HR}} (1 - q_0 q_1) + \frac{\mathit{HR}}{2 + \mathit{HR}}(1 - q_0^2 q_1) \Big) \\
\\
\end{eqnarray*}
\end{frame}

\begin{frame} %slide 14
\frametitle{$\pi_{t3}$}
\begin{eqnarray*} 
\pi_{t3} & = & \frac{1}{p_0 p_1^2}\Big(p_0 p_1^2 - 2 p_1 \int_0^\tau \lambda_0 e^{-\lambda_0 u}(1 - e^{-\lambda_1 u})du \\
&   & \phantom{p_0 p_1^2} + \int_0^\tau \lambda_0 e^{-\lambda_0 u}(1 - e^{-\lambda_1 u})^2du \Big)\\
& = & \frac{1}{p_0 p_1^2}\Big(p_0 p_1^2 - 2p_1(p_0 + \frac{\lambda_0}{\lambda_0 + \lambda_1}(e^{-(\lambda_0 + \lambda_1)\tau} - 1)) \\
&   & \phantom{p_0 p_1^2} + p_0 + \frac{2\lambda_0}{\lambda_0 + \lambda_1}(e^{-(\lambda_0 + \lambda_1)\tau} - 1) \\
&   & \phantom{p_0 p_1^2} - \frac{\lambda_0}{\lambda_0 + 2\lambda_1}(e^{-(\lambda_0 + 2\lambda_1)\tau}) - 1) \Big) \\
& = & \frac{1}{p_0 p_1^2}\Big( p_0 p_1^2 - 2 p_0 p_1 + p_0 + \frac{2}{1 + \mathit{HR}} (1 - p_1) (q_0 q_1 - 1) \\
&   & \phantom{p_0 p_1^2} - \frac{1}{2 + \mathit{HR}}(q_0 q_1^2 - 1) \Big) \\
& = & \frac{1}{p_0 p_1^2}\Big( p_0 q_1^2 + \frac{2}{1 + \mathit{HR}}q_1(q_0 q_1 - 1) - \frac{1}{2 + \mathit{HR}}(q_0 q_1^2 - 1) \Big) \\
\\
\end{eqnarray*}
\end{frame}

\begin{frame} %slide 15
\frametitle{$\pi_{x1}, \pi_{x2}$ and $\pi_{x3}$}
For $\pi_{x1}, \pi_{x2}$ and $\pi_{x3}$ note that
$X_{0k} \sim \mathcal{N}(\mu_0, \sigma^2)$, $X_{1l} \sim \mathcal{N}(\mu_1, \sigma^2)$ and $\delta = \mu_1 - \mu_0$ and hence

$Y_{0k}:=\frac{X_{0k} - \mu_0}{\sigma} \sim \mathcal{N}(0, 1)$ and $Y_{1l}:=\frac{X_{1l} - \mu_0}{\sigma} = \frac{X_{il} - \mu_1 + \delta}{\sigma} \sim \mathcal{N}( \frac{\delta}{\sigma}, 1)$
\\
Next step: Express $\pi_{x1}$, $\pi_{x2}$ and $\pi_{x3}$ in terms of the pdf, cdf of the standard normal distribution.

The following derivation actually holds for shift alternatives - not just for the normal distribution.
\end{frame}

\begin{frame} %slide 15
\frametitle{$\pi_{x1}, \pi_{x2}$ and $\pi_{x3}$}
\begin{eqnarray*}
\pi_{x1} & = & P(X_{0k} < X_{1l}) = P(Y_{0k} < Y_{1l}) = \int_{-\infty}^{\infty}\varphi(x - \frac{\delta}{\sigma}) \Phi(x) dx \\
\\
\pi_{x2} & = & P(\max(X_{0k}, X_{0k'}) < X_{1l}) = P(\max(Y_{0k}, Y_{0k'}) < Y_{1l}) \\
& = & \int_{-\infty}^{\infty}\varphi(x - \frac{\delta}{\sigma})\Phi^2(x) dx\\
\pi_{x3} & = & P(X_{0k} < \min(X_{1l}, X_{1l'})) = P(Y_{0k} < \min(Y_{1l}, Y_{1l'})) \\
& = & \int_{-\infty}^{\infty}\Phi(x) \cdot 2 \varphi(x - \frac{\delta}{\sigma}) (1 - \Phi(\frac{\delta}{\sigma}))dx\\
& = & \int_{-\infty}^{\infty}\varphi(x - \frac{\delta}{\sigma}) \Phi(x) dx - \Big( \Phi^2(x - \frac{\delta}{\sigma})\Phi(x)|_{-\infty}^{\infty}\\
&   & - \int_{-\infty}^{\infty} \Phi^2(x - \frac{\delta}{\sigma}) \varphi(x) dx\Big)\\
& = & 2\pi_{x1} - 1 + \int_{-\infty}^{\infty} \Phi^2(x - \frac{\delta}{\sigma}) \varphi(x) dx \\
\\
\end{eqnarray*}

\end{frame}
\end{document}
